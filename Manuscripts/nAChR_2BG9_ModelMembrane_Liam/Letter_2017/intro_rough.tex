% !TEX root = main.tex

The nicotinic acetylcholine receptor (nAChR) is an excitatory pentameric ligand gated ion channel (pLGIC) commonly found in the neuronal post synaptic membrane and neuromuscular junction (NMJ) in mammals %\cite{Breckenridge_Adult_1973,Cotman_Lipid_1969}, 
as well as the electric organs of the \textit{Torpedo} electric ray. %\cite{Bara89,Quesada_Uncovering_2016} . 
%nAChR is found at especially high concentrations (about $10^4$ $\mu m^{-2}$) within the NMJ membrane \cite{Breckenrldge1972}. 
%The pLGIC super family has been shown to play roles in cognition \cite{Walstab2010}, inflammation \cite{Patel2017,Yocum2017,Cornelison2016}, addiction \cite{Cornelison2016}, chronic pain \cite{Xiong2012} and numerous diseases including: Alzheimer's Disease, spinal muscular atrophy, and neurological autoimmune disease \cite{MartinRuiz_4_1999, Arnold_Reduced_2004,Lennon_Immunization_2003,Papke_The_2012,Picciotto_Neuroprotection_2008}.\grace{this sentence and citations are weird, aren't inflammation and addiction etc diseases?} nAChR plays a major role in excitation of the central and peripheral nervous system and binds agonists such as nicotine, acetylcholine and general anesthetics in multiple sites \cite{Bondarenko_NMR_2013,Jayakar_Identification_2013,LeBard_General_2012,Brannigan_Multiple_2010}.\graceforgrace{fix references here}
nAChRs play a fundamental role in rapid excitation within the central and peripheral nervous system, and neuronal nAChRs are also critical for cognition and memory \cite{Dani2001b, Changeux2015}. Acetylcholine is the orthosteric \nachr~ligand, but numerous other exogenous and endogenous small molecules modulate \nachr s, including nicotine, general anesthetics, the tipped-arrow poison curare,  phospholipids, cholesterol, and cholesterol-derived hormones.\cite{Klaassen2015,Taly2009}  %bind binds agonists such as nicotine, acetylcholine and general anesthetics in multiple sites \cite{Bondarenko_NMR_2013,Jayakar_Identification_2013,LeBard_General_2012,Brannigan_Multiple_2010}.\graceforgrace{fix references here} 
The larger pLGIC super family that includes \nachr s has been shown to play roles in numerous diseases related to inflammation, \cite{Patel2017,Yocum2017,Cornelison2016}, addiction \cite{Cornelison2016}, chronic pain \cite{Xiong2012}, Alzheimer's Disease \cite{Walstab2010,Picciotto_Neuroprotection_2008,MartinRuiz_4_1999}, spinal muscular atrophy \cite{Arnold_Reduced_2004}, schizophrenia \cite{Haydar2010} and neurological autoimmune diseases \cite{Lennon_Immunization_2003}.

nAChRs are highly sensitive to the surrounding lipid environment\cite{Hamouda2006a,Baenziger2017,Padilla-Morales2016,Barrantes2007} for reasons that remain poorly understood. In 1980 it was observed that \nachr s only conduct at native levels if model phospholipid membranes contain 10-20\% cholesterol
\cite{Dalziel1980,Criado1982,Ochoa1983}. Three generations of investigation have followed, with the first generation of studies\cite{Marsh1978,Dalziel1980,Marsh1981,Criado1982,Gonzalez-Ros1982,McNamee1982,Ellena1983,Ochoa1983,Zabrecky1985,Bristow1987,Leibel1987,Middlemas1987,Jones1988a,Jones1988, Fong1986,Fong1987,McNamee1988, Barrantes1989,Sunshine1992,Sunshine1994,Narayanaswami1993,Addona1998,Corbin1998,Barrantes2000} focused on differentiating between the role of bulk, annular, and non-annular cholesterol, a second generation\cite{Baenziger2015,Bruses2001,Marchand2002,Oshikawa2003,Pato2008,Zhu2006,Baenziger2017, Barrantes2007,Barrantes2000,Barrantes2010,Bermudez2010,Perillo2016,Wenz2005,Borroni2016, Unwin2017} probing membrane-mediated effects on organization of multiple \nachr s, and a third generation\cite{Basak2017,Althoff2014,Laverty2017,Zhu2018} applying x-ray crystallography and high-resolution cryo electron microscopy to directly observe lipid binding modes. 

Members of the pLGIC family other than \nachr~are also lipid-sensitive,\cite{Dunn1989,Sooksawate2001,Baenziger2011, Dostalova2014} and lipids other than cholesterol can also modulate function\cite{Bhushan1993,Cheng2007,Corrie2002a,DaCosta2002,Rankin1997,Wenz2005,Hamouda2006a}, but these mechanisms have not been as extensively studied.  The recent publication of several crystal and cryo-EM structures \cite{Basak2017,Althoff2014,Laverty2017,Zhu2018}
has confirmed that specific lipid-pLGIC interactions extend beyond cholesterol and \nachr.  Such interactions are also well-established in other transmembrane proteins, including G-protein coupled receptors (GPCRs) and other ion channels, as reviewed in \cite{Burger2000, Lee2004, Pucadyil2006a, Landreh2016, Smithers2012}. 

Even in the specific case of cholesterol-\nachr~ interactions, results from different approaches have suggested complex behavior and even contradictory interpretations.  Results have indicated that both cholesterol encrichment\cite{Dalziel1980,Criado1982,Ochoa1983} and cholesterol depletion\cite{Santiago2001} cause gain of function, that anionic phospholipids are unnecessary for native function\cite{Dalziel1980,Criado1982,Ochoa1983} or must be\cite{Corrie2002a,DaCosta2002} included in a reconstitution mixture,  that cholesterol increases \nachr~clustering\cite{Pato2008, Zhu2006, Barrantes2007} and directly interacts with \nachr~\cite{Leibel1987,Jones1988}, but \nachr~does not consistently partition into cholesterol-rich domains\cite{Bermdez_Partition_2010}. We suggest here that some of these apparent contradictions may be explained by competition between cholesterol and other lipids found in native membranes, primarily lipids with polyunsaturated fatty acyl chains (PUFAs).   

Interactions of \nachr~ with PUFAs have not been systematically investigated experimentally, but a large amount of circumstantial experimental evidence suggests an important role for PUFAs in \nachr~ function.    Clinically, long-chain $n-3$  (commonly called ``Omega-3'' or $\omega-3$) lipids have a neuroprotective role\cite{Piomelli2007}, and \nachr-associated pathologies can arise for patients with low levels of $n-3$ PUFAs. $\alpha7$ \nachr s are implicated in schizophrenia\cite{Haydar2010}, and dietary supplementation with $n-3$ fatty acids (usually through fish oil) can reduce the likelihood of psychosis, with dramatic effects in some individual cases.\cite{Amminger2010}   

{\it In vitro}, PUFA-rich asolectin\cite{Regost2003,Olsen2003} is one of the most robust additives\cite{Criado1982} for obtaining native \nachr~function. The specific component(s) of asolectin responsible for conferring function have not been identified. 
%The dearth of data is likely due the fact working with polyunsaturated fatty acids (PUFAs) in the laboratory introduces challenges due to lipid peroxidation.  Nonetheless, 
Long chained $n-3$  (previously termed $\omega-3$)  PUFA lipids are abundant in two seemingly disparate \nachr~native membranes: mammalian neuronal membranes\cite{Breckenridge_Adult_1973,Cotman_Lipid_1969} and those of the \textit{Torpedo} electric organ,\cite{Barrantes1989,Quesada2016}. Both such membranes also have an abundance of phosphoethanolamine (PE) headgroups and saturated glycerophospholipids, and a scarcity of monounsaturated acyl chains and sphingomyelin compared to \textit{Xenopus} oocyte membranes \cite{Hill_Isolation_2005} common in functional studies, or even a ``generalized'' mammalian cell membrane \cite{Inglfsson_Lipid_2014}.    % and has been shown to be functionally dependent on cholesterol and anionic lipids when reconstituted into a membrane \cite{Fong_Correlation_1986,Sunshine_Lipid_1992,Hamouda_Assessing_2006,Butler_FTIR_1993,Bhushan_Correlation_1993,Fong_Stabilization_1987,Bednarczyk_Transmembrane_2002,Corrie_Lipid_2002}. The absence of native cholesterol concentrations does not inhibit ligand-nAChR binding, but does prevent gating \cite{Baenziger2015,Carswell_Role_2015,Calimet2013}, impeding ion flux through the pore \cite{Fong_Correlation_1986,Sunshine_Lipid_1992,Hamouda_Assessing_2006,Butler_FTIR_1993,Bhushan_Correlation_1993,Fong_Stabilization_1987,Bednarczyk_Transmembrane_2002,Corrie_Lipid_2002,Cheng_Anionic_2009}. \grace{references} Previous research suggests cholesterol may be bound within the inter- and intra-subunits of the transmembrane domain (TMD) \cite{Brannigan_Embedded_2008}; and cholesterol has been hypothesized and recently found bound within the $\gamma$-Aminobutyric acid receptors (GABAARs) TMD \cite{Hnin_A_2014,Laverty2017}. \grace{add new Hibbs citation} 
%\grace{The first and second generation of studies primarily focused on cholesterol, while the third generation has been focused on any lipids which are resolved in high resolution structures, including cholesterol or phospholipids.  }
% Adding cholesterol to the model membrane increases the ability of the channel to gate, with
%an EC$_{50}$ of 10\% cholesterol and a Hill coefficient of 2 \cite{Rankin1997}.  
%Results suggesting direct interactions between cholesterol and the \nachr, included (1) a pool of cholesterol that cannot be depleted from native Torpedo \nachr-rich membranes,\cite{Leibel1987} (2) quenching of intrinsic nAChR fluorescence by bromocholesterol \cite{Jones1988}  (3)  \nachr-local boundary lipids with rotational correlation times an order of magnitude slower than those in the bulk %with the spin label 14-PCSL, %exchanged with the bilayer at a rate of 6 $\times 10^7$s$^{-1}$  

%demonstrated specific binding of lipids to pLGICs, including Didocosahexaenoyl (DHA) bound Gloeobacter Ligand-gated Ion Channel (GLIC)\cite{Basak2017}, Palmitoyloleoylphosphatidylcholine (POPC) bound Glutamate-gated Chloride Channel (GluCl)\cite{Althoff2014}, and cholesterol-hemisuccinate bound  $\gamma$-aminobutyric acid (A) (GABA(A)) receptor\cite{Laverty2017,Zhu2018}\liam{. In} some cases\cite{Althoff2014, Zhu2018} density deep within the TMD bundle is assigned to phospholipid headgroups\cite{Althoff2014} or cholesterol hemisuccinate\cite{Zhu2018}.   

% and (4) spin labeled lipids indicated the steroid had a higher affinity for \nachr~ than monounsaturated phosphatidylcholine (DOPC) \cite{Abadji1994}.  
%This titration agrees well with the . 
%Action of general anesthetics on the nAChR that were present in native membranes or cholesterol-containing bilayers were absent in the model membranes lacking cholesterol, probably because cholesterol was required to support the necessary conformation change \cite{Raines1994};
 % I've always been curious about the data for this abstract (I didn't find it in a ms).
%a.	Shades of John Baenziger?s uncoupled state
%The action of cholesterol on nAChRs was independent of its ability to flip?flop across the bilayer, but was consistent with a binding site near, but not in, the monounsaturated lipid-protein interface \cite{Addona1998}. 
% Steroids of different stereochemistry supported nAChR activation independent of structural features known to be important for modulation of lipid bilayer properties \cite{Addona2003}.
%A boundary  lipid local to the \nachr, probed with the spin label 14-PCSL, exchanged with the bilayer at a rate of 6 $\times 10^7$s$^{-1}$ and had a rotational correlation time an order of magnitude slower than when in the bilayer.\cite{Abadji1993}.
%\end{enumerate}
%I made a few changes to the above to make them more precise with regard to saturation, please let me know if anything is incorrect. 
%These studies primarily focused on how cholesterol interacted with \nachrz. %Agonist induced rapid fluorescence quenching on the millisecond time scale. Pharmacological criteria were used to establish the specificity of the flux. In a given set of vesicles, the amplitude of flux depended on agonist concentration and a concentration-response curve could be obtained. However, the absolute amplitude of flux could not be compared between preparations presumably because of vesicle morphology issues. 
%Measurable ion flux for reconstituted \gabaar was obtained using vesicles containing asolectin:brain (4:1), but not with asolectin alone.\cite{Dunn1989} No systematic study of the effects of lipid composition was made, even in a followup study;\cite{Dunn1994} it is now established that brain lipids contain primarily $\omega-3$ PUFAs, while asolectin contains $\omega-6$ PUFAs.  
%{Furthermore, most such data was collected and interpreted in an era with little information on \plgic structure or awareness of lateral organization in membranes. }  %In this proposal,  {\bf ``boundary''} lipids refers to lipids bound directly to the \plgic, in either annular or non-annular sites, although the latter may be deep within the protein.  
%Nonetheless, PUFA-rich asolectin\cite{asolectin} has served as the most reliable\cite{which?} reconstitution mixture thus far. 

%%%%%%

%Systematic investigations of the mechanism of \nachr~modulation by lipids have primarily focused on cholesterol, with minimal focus on the effects of polyunsaturation-pLGIC interaction. \liam{There have been a number of studies on PUFA ion channel interaction \cite{Landreh2015,Landreh2016,Caires2017,Borjesson2008,Leaf2002,Xiao2005}, however, these channels are not pLGICs. %The incorporation of asolectin, which is rich in PUFAs, into biological models containing \nachr~in order to restore function \cite{RyanSEDemersCNChewJP1996,DaCosta2005,Criado1984}
%The dearth of data} is likely due the fact working with polyunsaturated fatty acids (PUFAs) in the laboratory introduces challenges due to lipid peroxidation.  Nonetheless, long chained $n-3$  (previously termed $\omega-3$)  PUFA lipids serve as a distinctive common element of two seemingly disparate \nachr~native membranes: mammalian neuronal membranes\cite{Breckenridge_Adult_1973,Cotman_Lipid_1969} and those of the \textit{Torpedo} electric organ,\cite{Barrantes1989,Quesada2016}. Both such membranes also have an abundance of phosphoethanolamine (PE) headgroups and saturated glycerophospholipids, and a scarcity of monounsaturated acyl chains and sphingomyelin compared to \textit{Xenopus} oocyte membranes \cite{Hill_Isolation_2005} common in functional studies, or even a ``generalized'' mammalian cell membrane \cite{Inglfsson_Lipid_2014}. PUFA-rich asolectin\cite{Regost2003,Olsen2003} is one of the most robust additives\cite{Criado1982} for obtaining native \nachr~function, but the specific component has not been identified. Finally, long-chain $n-3$ lipids have a neuroprotective role,\cite{Piomelli2007} and supplementation with $n-3$ lipids can significantly reduce the likelihood of psychosis in schizophrenia-susceptible patients.\cite{Amminger2010}   
%It is now established that brain and \textit{Torpedo} membrane PUFAs primarily have $\omega-3$ chains, while asolectin.  
%Native membranes for \nachr s, including post-synaptic membranes and those of the \textit{Torpedo} electric organ, are enriched in phosphoethanolomine (PE) and  $\omega-3$ polyunsaturated fatty acids (PUFAs)\cite{Breckenridge_Adult_1973,Cotman_Lipid_1969,Bara89,Quesada_Uncovering_2016}, especially . 

 %The \liam{\textit{Torpedo} electric organ's membrane phospholipid composition \cite{Bara89,Quesada_Uncovering_2016} is comparable to the  post-synaptic membranes}. \% are similar in lipid composition to the electric organs of \textit{Torpedo} \cite{Bara89,Quesada_Uncovering_2016}.  

Membranes composed of ternary mixtures of saturated lipids, unsaturated lipids, and cholesterol tend to demix into separate domains. Saturated lipids and cholesterol constitute a rigid liquid ordered phase ($\lo$) in which acyl chains remain relatively straight. \liam{ \cite{Feller_Acyl_2008,Yeagle2016115,Cicuta1981,Bleecker2016}} Unsaturated lipids form a more flexible liquid disordered phase ($\ldo$) in which the chains remain fully melted.  %Domain formation has been studied both experimentally and computationally in model membranes \cite{Lingwood_Lipid_2010,Kaiser_Order_2009,Ma_n_2004,Inglfsson_Lipid_2014,Risselada_The_2008}, showing de-mixing of lipids with saturated fatty acids and unsaturated fatty acids \cite{Levental_Polyunsaturated_2016,Lor2015}. 
%Monounsaturated fatty acids contain a single double bond, while PUFA's contain two or more double bonds through their acyl chain. These double bonds make unsaturated lipids both disordered and highly flexible \cite{Lingwood_Lipid_2010,Pato_Role_2008,Risselada_The_2008,Schley_2007,Rawicz_Effect_2000}. 
%PE (a zwitterionic head group) is one of two major head groups, the other being phosphocholine (PC). nAChR-lipid studies using model membranes with zwitterionic head groups have not included PUFAs, instead favoring saturated and monounsaturated fatty acids. %need proper refs
$\lo$ domains are often visualized as signaling ``platforms'', restricting membrane proteins into a high density ``raft'' that travel around the fluid membrane \liam{\cite{Simons1997,Simons2000}}. 

The first generation of studies into the mechanism underlying cholesterol-modulation of \nachr~ were conducted and interpreted in an era preceding the discovery of lipid-induced domain formation in membranes. %Domain formation is as sensitive to phospholipid unsaturation as it is to cholesterol content. 
The second generation explicitly considered potential interactions of \nachr~with lipid domains.  Since direct interaction between \nachr~and cholesterol had been demonstrated in the first generation of studies, a sensible initial hypothesis was that \nachr~persistently partitioned to $\lo$ domains, retaining little contact with unsaturated chains. Conclusive support for this hypothesis has not been found.  %Density and aggregation of \nachr s in cellular membranes is sensitive to cholesterol depletion. 

Barrantes and colleagues\cite{Wenz2005} found that the addition of \nachr~to a domain-forming lipid mixture increased the size of Dipalmitoylphosphatidylcholine/Cholesterol (DPPC/Chol) lipid-ordered domains, which (combined with additional FRET data) was interpreted as indicating \nachr~was embedded in liquid-ordered domains.  Other experiments investigating whether \nachr s partition into lipid microdomains have been largely contradictory. Some\cite{Marchand2002,Stetzkowski-Marden2006,Willmann2006} suggest that \nachr s are associated with microdomains independently of stimulation by other proteins. Other studies\cite{Zhu2006,Campagna2006} found that \nachr s require stimulation by a protein such as agrin to partition into microdomains. Formation and disassembly of the \nachr-rich microdomains is highly sensitive to cholesterol concentration,\cite{Barrantes2007,Bruses2001,Marchand2002,Zhu2006,Pun2002}

These studies suggested a role for cholesterol-induced phase separation, but do not confirm that \nachr~partitions to the cholesterol-rich phase.  To test for an intrinsic \nachr~ domain preference, Barrantes and co-workers checked for enrichment of \nachr s in the detergent resistant membrane (DRM).   \nachr s were not enriched in the DRM of a model, domain-forming mixture (1:1:1  Chol:POPC:sphingomyelin) \cite{Bermdez_Partition_2010} but inducing compositional asymmetry across leaflets did yield \nachr~enrichment in the DRM fraction \cite{Perillo_Transbilayer_2016}.  While more precise and robust experimental methods for determining partitioning preference and specific boundary lipids such as mass spectrometry have been applied for other transmembrane proteins\cite{Gupta2018,Chorev2018}, they have not been applied to complex heteromers like \nachr.  

Fully atomistic molecular dynamics (MD) simulations\cite{Brannigan_Embedded_2008, Cheng2009, Hnin_A_2014, Carswell_Role_2015 } have served as a natural complement to the third-generation structural biology approach, but are limited in their ability to resolve contradictions between first and second generation studies, because lipids are unable to diffuse over simulation time scales.\cite{Ingolfsson2014,Bond2006,Parton2013,Goose2013,Scott2008}.   Efficient lipid diffusion is a requirement for equilibrating domains or detecting protein-induced lipid sorting.    Coarse-grained MD (CG-MD) has been used to great success in a number of simulations for both lipid-protein binding and membrane organization \cite{Bond2006,Scott2008,Parton2013,Goose2013,Iyer2018,Sodt2014}. Here we use this method as a ``computational microscope'' to observe the equilibrium distribution of lipids local to the \nachr~ in a range of binary and ternary lipid mixtures inspired by native membranes.   We observe a remarkable enrichment of polyunsaturated lipids among \nachr~boundary lipids. To our knowledge, these are the first molecular simulations of the \nachr~in non-randomly mixed membranes, and the first study to systematically investigate the likelihood of polyunsaturated lipids as \nachr~boundary lipids. 
% !TEX root = main.tex

In this work we used coarse-grained simulations to predict the local lipid composition around the nicotinic acetylcholine receptor, in a range of domain forming membranes.  We observed  \nachr~partitioning to the liquid-disordered phase in all systems for which such a phase was present.  This is inconsistent with the model of lipid rafts as platforms that contain a high density of nAChRs, and unexpected in light of the established cholesterol dependence of nAChR.  As shown in these simulations, partitioning to the $\ldo$ phase does not prevent \nachr~from accessing cholesterol. 

The simulations presented here involve only one receptor per system.  Using the present results only, the simplest extrapolation to multiple receptors would assume that receptors are simply distributed randomly across the $\ldo$ domain.  The local receptor area density would be the number of receptors divided by the total area of $\ldo$ domains.    

In the model membranes used here, as well as in native nAChR membranes, the lipid composition would be expected to yield $\ldo$ phases that were about the same size as $\lo$ phases. The $\lo$ ``raft'' in an $\ldo$ ``sea'' analogy is not representative when over 50\% of the membrane is in the ``raft'' phase.  A more representative analogy would be receptors as boats, floating on an $\ldo$ lake within an $\lo$ rigid land mass.  Filling in the lake by adding to the coastline would force any boats in the lake closer together.  Similarly, any process that decreased total $\ldo$ area while keeping the number of receptors constant would increase the local receptor density.   Observing increased \nachr~density by adding membrane cholesterol (as in \cite{%Fernandes2010,
Barrantes2014,Rodrigues2013,Bruses2001,Marchand2002,Oshikawa2003,Pato2008,Zhu2006, Barrantes2007,%Barrantes2000,
Barrantes2010,Wenz2005,Borroni2016})\grace{just checking, did these all show an INCREASE in density and/or clustering with an increase in cholesterol?} would thus be consistent with \nachr~partitioning to the cholesterol-poor phase. 

This extrapolation assumes that introduction of additional receptors does not change partitioning behavior.  We do still find reliable partitioning to the $\ldo$ phase upon adding more receptors, and we will characterize systems with multiple receptors in a future publication. Due to receptor dimerization and trimerization, distribution of individual receptors within the $\ldo$ phase will not be random.   This would not change the expected trend of density increasing with added cholesterol, however.  

Observed partitioning into the $\ldo$ phase could be considered inconsistent with interpretations of some experiments, \cite{Bermdez_Partition_2010,Perillo_Transbilayer_2016}	which suggest minimal nAChR partitioning preference in symmetric model membranes or an actual preference for an $\lo$ phase in asymmetric model membranes.  These experiments used only monounsaturated acyl chains, and may have had less well-defined domains.  They further relied on detergent resistant membrane (DRM) methods, which are sensitive to {the} choice of detergent \cite{Brown2007} and could be unable to distinguish between proteins with no partitioning preference vs proteins that persistently partition to one side of a boundary. 

%\liam{Computationally, S. Iyer et al. and A Sodt et al. \cite{Iyer2018,Sodt2014} suggest that cholesterol may act as an interfacial lipid in simulations using the monounsaturated lipid DOPC. This is not observed to hold in ternary mixtures including DHA, cholesterol and DOPC or POPC. In Figure SI\ref{fig:OL}, cholesterol is observed to mix readily with DOPC and POPC if DHA is present. This result maybe the resultant of the MARTINI DOPC parameter not properly reflecting its atomistic counterpart. \cite{Carpenter2018}}

%\liam{There have been a number of studies measuring synaptic clustering \cite{Fernandes2010,Barrantes2014,Rodrigues2013,Pick2003}, relating the theory, the more cholesterol the more likely the $\lo$ phase is for \nachr. Our results, suggesting \nachr~is in the $\ldo$ phase, reasons against this hypothesis. With an increased $\lo$ lipid concentration, and a decreased $\ldo$ concentration may suggest a smaller domain for \nachr~clustering, rusting in a greater \nachr~ density.}
	
The origin of preferential partitioning observed in these simulations for the $\ldo$ domain is still unclear, but may reflect different elastic properties of the $\ldo$ and $\lo$ domains.  In general, proteins embedded in membranes will introduce a boundary condition on the membrane shape, such that (1) the thickness of the membrane matches the thickness of the transmembrane domain\cite{Aranda-Espinoza1996, Jensen2004, Brannigan2006} and (2) interfacial lipids are parallel to the protein surface.\cite{goulian1993}.  Transmembrane proteins with hydrophobic mismatch with the surrounding membrane may deform the membrane thickness to satisfy constraint (1), while cone-shaped proteins like pLGICs~ must also introduce a ``tilt'' deformation to satisfy (2).  Each leaflet of the membrane has an elastic resistance to bending away from its spontaneous curvature, and satisfying these constraints is energetically costly.  

Continuum theories based on the Helfrich Hamiltonian have been used to predict shape deformations around protein inclusions in homogeneous membranes.\cite{goulian1993,Aranda-Espinoza1996,Brannigan2006}  In mixed membranes, minimization of the protein-deformation free energy may also induce lipid sorting.  Two distinct sorting mechanisms could minimize the bending free energy: sorting that A) reduces the required bending deformation, by selecting boundary lipids with a specific thickness, leaflet asymmetry, or shape or B) reduces the free energy cost of the bending deformation, by selecting for flexible boundary lipids.   Mechanism (B) is the most generally applicable approach, and would stabilize partitioning to the most flexible domains, consistent with our observations (Figure SI\ref{fig:curve}).  In some cases, mechanism (A) may also contribute to partitioning or lipid-sorting, and could explain why \nachr~tends to attract saturated PE over saturated PC, or how leaflet asymmetry can promote partitioning to more rigid phases as observed in \cite{Perillo_Transbilayer_2016} . %\liam{As \nachr~will aggregate into densities of about $10^4$ $\mu m^{-2}$ at the post synaptic interface within the NMJ \cite{Breckenrldge1972}, the effect of the conical shape and protein clustering will undoubtedly play a role in both membrane organization and protein clustering.}
%\liam{Simulations predict nAChR partitioning into $\ldo$ domains. We initially hypothesized subunit-lipid preferenciation, however it is not consistently observed in current simulations.}

We previously \cite{Brannigan_Embedded_2008} proposed unresolved density in the cryo-EM structure of \nachr~ in the {\it Torpedo} membrane could be embedded cholesterol, based on gain of function caused by cholesterol in reconstitution mixtures\cite{Fong_Correlation_1986,Sunshine_Lipid_1992,Hamouda_Assessing_2006,Butler_FTIR_1993,Bhushan_Correlation_1993,Fong_Stabilization_1987,Bednarczyk_Transmembrane_2002,Corrie_Lipid_2002}, but we did not consider the possibility of occupation by polyunsaturated chains.  %Restoration of native function by cholesteryl hemisuccinate (CHS) is observed only over a narrow CHS concentration range in monounsaturated PE/PS membranes, but a much wider range in asolectin\cite{Criado1982}, which is particularly rich in $n-6$ PUFAs.  
Here we observe spontaneous binding of cholesterol to coarse-grained embedded sites, but long-chain PUFA tails displace cholesterol in some binding sites. Long acyl chains may penetrate far into the TMD bundle without requiring the entire head group also be incorporated, and long-chain PUFAs may do so without as substantial an entropic penalty as long saturated chains.  Cholesterol (like phosphatidic acid, another lipid known to cause gain of function under some preparations\cite{Butler_FTIR_1993,Bhushan_Correlation_1993,Fong_Stabilization_1987,Bednarczyk_Transmembrane_2002,Corrie_Lipid_2002})  has a much smaller headgroup than PC or PE. It can become fully incorporated into the TMD without the TMD needing to accommodate the bulky headgroup.   These complex associations underlie the challenges of predicting local lipid environment in heterogeneous, highly non-ideal mixtures. 

%\liam{Brannigan et al. 2008 \cite{Brannigan_Embedded_2008} hypothesized the density gaps in the nAChR cryo-EM structure (PDB 2BG9) \cite{Unwin_Refined_2005}, nAChR's gaps were embedded with cholesterol, rather than the initially proposed water. These simulations show embedded cholesterol through nAChR's TMD, however, there tends to be more DHA embedded within the TMD than cholesterol.}

%Our simulation results consistently predict PUFAs and cholesterol to embed throughout nAChR. Cholesterol, however does not reside in the $\ldo$ phase. The sporadic nAChR-domain boundary interactions observed, however, may suggest may suggest a pathway for access of cholesterol for nAChR-sites, despite nAChR partitioning into the cholesterol-poor $\ldo$ phase .
%
%\liam{Multiple sources such as \cite{Sunshine_Lipid_1992,Hamouda_Assessing_2006,Bhushan_Correlation_1993,Fong_Stabilization_1987,Corrie_Lipid_2002} suggest} anionic lipid head groups have greater specificity for nAChR than zwitterionic head groups; the head group phosphatidic acid has been shown to improve nAChR functionality \cite{Butler_FTIR_1993,Bhushan_Correlation_1993,Fong_Stabilization_1987,Bednarczyk_Transmembrane_2002,Corrie_Lipid_2002}. Anionic lipids must be included in future simulations to better ascertain if nAChR has a greater preference for PUFAs or charged head groups.

%We hypothesize PUFAs flexibility allows them to take on multiple confirmations, assisting with deep non-annular binding. Both DHA and LA are observed to embed in nAChR, though LA appears to not embed as deeply or as frequently as nAChR. We hypothesize saturated lipids chain's rigidity hinders embedding. 

All simulations reported here contain lipids with di-saturated tails or di-PUFA tails. While lipid species with two identical acyl chains do exist in the native membrane, they are far less common than hybrid lipids with heterogeneous acyl chains.  Including hybrid lipids would reduce the potential for formation of large domains, while increasing the length of the domain interface. 
Incorporation of hybrid lipids would also reduce the \nachr-local concentration of PUFA chains.  Even 5-10\% DHA is a saturating concentration for \nachr cavities, however, so we expect occupation of cavities to be minimally affected by replacement of di-DHA lipids with twice the number of hybrid lipids. 

%Boundary lipids for a transmembrane protein must interact favorably with the protein, but must also interact favorably with other boundary lipids.  Replacing saturated lipids with cholesterol reduces depletion of saturated lipids in the \nachr~boundary when domains are present but not when they're absent, which suggests \nachr~will accommodate the saturated lipid that must be near cholesterol in a domain separated system. This could be consistent with cholesterol acting as a surfactant between \nachr~and a well-defined $\lo$ domain, and/or \nachr~acting as a surfactant between the $\ldo$ domain and $\lo$ domain. Similarly, we observe that occupation of the $\beta$ subunit interior by cholesterol is actually increased by an $\ldo$ annulus, likely because cholesterol rings can be fully sequestered from the $\ldo$ domain within the spacious $\beta$ subunit interior.
%\liam{This research has not delved into the  analysis of the effect nAChR's plays on local and global membrane deformations. Our analysis suggest smaller box sizes $(25x25 nm^2)$ allow for the protein to effect membrane undulations across the periodic boundary causing artifacts along the water box's axis. Initial simulations show larger membranes limit this artifact.} 



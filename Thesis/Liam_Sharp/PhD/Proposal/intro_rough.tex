The nicotinic acetylcholine receptor (nAChR) is an excitatory pentameric ligand gated ion channel (pLGIC) commonly found in the post synaptic membrane, neuromuscular junction (NMJ) \cite{Breckenridge_Adult_1973,Cotman_Lipid_1969} and the \textit{Torpedo} electric organ \cite{8DAAC844-CF26-B5A6-FE56-AEE1F681B8A3,Quesada_Uncovering_2016} . nAChR is found at concentrations around $10^4$ $\mu m^{-2}$ within the NMJ membrane \cite{Breckenrldge1972}. 

The pLGIC super family has been shown to play roles in cognition \cite{Walstab2010}, inflammation \cite{Patel2017,Yocum2017,Cornelison2016}, addiction \cite{Cornelison2016}, chronic pain \cite{Xiong2012} and numerous diseases including: Alzheimer's Disease, spinal muscular atrophy, and neurological autoimmune disease \cite{MartinRuiz_4_1999, Arnold_Reduced_2004,Lennon_Immunization_2003,Papke_The_2012,Picciotto_Neuroprotection_2008}. nAChR plays a major role in excitation of the central and peripheral nervous system and binds agonists such as nicotine, acetylcholine and general anesthetics in multiple sites \cite{Bondarenko_NMR_2013,Jayakar_Identification_2013,LeBard_General_2012,Brannigan_Multiple_2010}.

nAChR is highly lipid sensitive and is functionally dependent on cholesterol and anionic lipids when reconstituted into a membrane \cite{Fong_Correlation_1986,Sunshine_Lipid_1992,Hamouda_Assessing_2006,Butler_FTIR_1993,Bhushan_Correlation_1993,Fong_Stabilization_1987,Bednarczyk_Transmembrane_2002,Corrie_Lipid_2002}. Lacking native-like concentrations of cholesterol or an abundance of anionic lipids in reconstituted membranes does not inhibit ligand-nAChR binding, but prevents gating and conformational changes \cite{Baenziger2015,Carswell_Role_2015,Calimet2013}, impeding ion flux through the pore \cite{Fong_Correlation_1986,Sunshine_Lipid_1992,Hamouda_Assessing_2006,Butler_FTIR_1993,Bhushan_Correlation_1993,Fong_Stabilization_1987,Bednarczyk_Transmembrane_2002,Corrie_Lipid_2002,Cheng_Anionic_2009}. Previous research suggests cholesterol may be bound within the inter- and intra-subunits of the transmembrane domain (TMD) \cite{Brannigan_Embedded_2008}; and cholesterol has been hypothesized and recently found bound within the $\gamma$-Aminobutyric acid receptors (GABAARs) TMD \cite{Hnin_A_2014,Laverty2017}. 

nAChR's native membranes (\textit{Torpedo}, synaptic)\cite{Breckenridge_Adult_1973,Cotman_Lipid_1969,8DAAC844-CF26-B5A6-FE56-AEE1F681B8A3,Quesada_Uncovering_2016}, are enriched in phosphoethanolomine (PE) and polyunsaturated fatty acids (PUFAs) when compared oocyte\cite{Hill_Isolation_2005}, and a generalized mammalian cell \cite{Inglfsson_Lipid_2014}. Unsaturated lipids are likely to form domains as unsaturated acyl chain's disorder prevents unsaturated lipids from easily mixing with the rigid saturated-, sphingolipids and cholesterol \cite{Feller_Acyl_2008,Yeagle2016115}; forming domains depleted in cholesterol, labeled liquid disordered ($l_{do}$) domains. Domain formation has been studied both experimentally and computationally in model membranes \cite{Lingwood_Lipid_2010,Kaiser_Order_2009,Ma_n_2004,Inglfsson_Lipid_2014,Risselada_The_2008}, showing de-mixing of lipids with saturated fatty acids and unsaturated fatty acids \cite{Levental_Polyunsaturated_2016,Lor2015}.

%Monounsaturated fatty acids contain a single double bond, while PUFA's contain two or more double bonds through their acyl chain. These double bonds make unsaturated lipids both disordered and highly flexible \cite{Lingwood_Lipid_2010,Pato_Role_2008,Risselada_The_2008,Schley_2007,Rawicz_Effect_2000}. 


%PE (a zwitterionic head group) is one of two major head groups, the other being phosphocholine (PC). nAChR-lipid studies using model membranes with zwitterionic head groups have not included PUFAs, instead favoring saturated and monounsaturated fatty acids. %need proper refs

nAChR's functional dependency on cholesterol has suggested that nAChR partitions into ordered cholesterol enriched domains \cite{Bermdez_Partition_2010,Perillo_Transbilayer_2016,Pato_Role_2008,Fong_Correlation_1986,Sunshine_Lipid_1992}, liquid ordered ($l_o$) domains. Bermdez et al \cite{Bermdez_Partition_2010}, showed nAChR to partitioned equally into $l_o$ and $l_{do}$ domains in model membranes of Chol:POPC:SM 1:1:1. Expanding on \cite{Bermdez_Partition_2010}, Perillo et al \cite{Perillo_Transbilayer_2016} showed using the previous composition but inducing asymmetrical membrane compositions, promoted nAChR to partition into the $l_o$ domain.% These results may represent nAChR sitting at an interface instead of in a $l_{do}$ domain. %Research predicting nAChR partitioning into  $l_o$ phase have used model membranes composed of cholesterol, lipids with saturated acyl chains, and lipids with anionic head groups (such as  Palmitoyloleoylphosphatidylcholine (POPC) and  Palmitoyloleoylphosphatidic acid  (POPA)). 

Through coarse grained molecular dynamics simulations, we have analyzed nAChR-lipid interaction within quasi-native membrane using the cryo-EM nAChR structure derived by Unwin in 2005 \cite{Unwin_Refined_2005}. The coarse grained model allows for significantly larger systems to be constructed using atomistic models. Running simulations over $\mu s$ allows systems to approach equilibrium, showing domain formation and protein partitioning. 

We use the PUFAs Docosahexaenoic acid (22:6 $n-3$) (DHA) and Linoleic acid (18:2 $n-6$) (LA). Nearly all the $n-3$ PUFAs in both synaptic and \textit{Torpedo}'s electric organ have been determined to be DHA \cite{Breckenridge_Adult_1973,Cotman_Lipid_1969,8DAAC844-CF26-B5A6-FE56-AEE1F681B8A3,Quesada_Uncovering_2016}. LA was a useful test fatty acid; Risselada et al \cite{Risselada_The_2008} showed it a usable PUFA for domain formation using Martini \cite{martini}.

Our simulations show embedded nAChR consistently partitions into the  $l_{do}$ phase. This research explores nAChR's partitioning behavior, boundary lipid affinity, and deep non-annular lipid-protein interactions termed ''embedding'' within domain forming membranes. It is our understanding this is the first study applying coarse grained molecular dynamics to nAChR in a membrane containing the most prominent native PUFA, DHA.
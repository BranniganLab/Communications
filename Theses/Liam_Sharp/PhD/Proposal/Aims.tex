Nicotinic acetylcholine receptors (nAChRs) are pentameric ligand gated ion channels (pLGICs) found through out the central and peripheral nervous system. Function of these neurotransmitter receptors is very sensitive to composition of the surrounding lipid membrane. However, the mechanisms of interaction between nAChR and its lipid environment are poorly understood. Research has shown a requirement for cholesterol in reconstitution mixtures of nAChR and other neuronal pLGICs, but other potentially essential lipids abundant in native membranes (particularly n−3 polyunsaturated fatty acids or PUFAs) have not been investigated. For my PhD research I propose the following:

\begin{enumerate}
  \item \textbf{Aim 1: Coarse-grained simulations of multiple subtypes of mammalian pLGICs in quasiphysiological membranes}  We observe nAChR partitioning into  n-3 polyunsaturated fatty acid (PUFA)  rich domains with the n-3 PUFA DHA-PE as nAChR's primary boundary lipid. The concentration of n-3 lipids is much lower in membranes, such as Xenopus oocytes,  commonly used in electrophysiology experiments, than in native membranes. I hypothesize adding small concentrations of n-3 is likely to restore the native boundary lipids. I will model various n-3 supplemented quasi-physiological membranes (such as oocytes) to predict those likely to provide a native local environment within the non-native membrane.%, for later testing by an experimental collaborator (Dr. John Baenziger of University of Ottawa). %We observe nAChR partitioning into n-3 (DHA) rich domains with DHA-PE as nAChR’s primary boundary lipid. Xenopus oocytes lipid composition are considerably different from neuronal membranes, but due to the high affinity of DHA chains for nAChR , a modest supplementation scheme is likely to restore the native boundary lipids. We will model various supplementation schemes to predict those likely to provide a native local environment within an oocyte, for later testing by an experimental collaborator (Dr. John Baenziger of University of Ottawa).

  \item \textbf{Aim 2: Investigation of the relative importance of pLGIC sequence vs shape in determining preferred lipid domain} This can be tested by comparing effects on partitioning profiles upon mutation of lipid facing residues versus adjustments in membrane lipid composition. If the effect of the protein's sequence is measured to be greater than its shape, it is likely that pLGICs will display significant variation in partitioning behavior and annular lipid preferences. If the reverse is observed, it is likely that overall pLGIC shape and relative flexibility of domains drives partitioning, and thus all pLGICs may have similar partitioning behavior.%This can be tested comparing effects on partitioning profiles upon mutation of lipid facing residues versus adjustments in membrane lipid composition. If the former has a stronger effect, it is likely that pLGICs will display significant variation in partitioning behavior and annular lipid preferences, while if the latter has a stronger effect, it is likely that overall pLGIC shape and relative flexibility of domains drives partitioning, with similar preferences across the pLGIC family.

  \item \textbf{Aim 3: Development and release of a user-friendly VMD plugin for measuring elastic parameters of heterogenous membranes} %%This package could be of significant use to both chemists and biophysicists. It would allow computational researchers to easily predict the elasticity moduli of a membrane composed of a general lipid mixture. 
  While multiple individuals have developed scripts to determine the fluctuation spectrum of membranes, there is no universal tool computational chemists and biophysicists can use. I will develop a tool to measure elastic parameters and fluctuation spectra within the convenient scripting environment of the VMD software, which will alleviate the daunting nature of solving for the fluctuation spectrum and related moduli.  This tool will assist us in optimizing lipid selections for modeled neuronal membranes; I can adjust lipid species and lipid concentrations to mimic elastic properties of neuronal membranes. 
\end{enumerate}

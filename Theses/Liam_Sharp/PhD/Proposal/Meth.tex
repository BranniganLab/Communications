% !TEX root = main.tex
\subsubsection{System Composition}

All simulations were built with Martini's martinize.py to coarse grain the PDB structure and insane.py to embed the protein within a coarse grained membrane \cite{martini}. nAChR coordinates came from PDB 2BG9 derived by Unwin et al. 2005 \cite{Unwin_Refined_2005}. 

The lipids used are Dipalmitoylphosphatidylcholine (DPPC), Dipalmitoylphosphatidylethanolamine(DPPE), Didocosahexaenoylphosphatidylethanolamine (DHA-PE), Didocosahexaenoylphosphatidylcholine (DHA-PC), Dilinoleoylphosphatidylcholine (DLiPC), Dilinoleoylphosphatidylethanolamine (DLiPE) and cholesterol (Chol).

Simulations were deigned to be composed of a saturated lipid, a PUFA, and Chol over concentrational three space. We focused on primarily two series of simulations: 1) systems with DPPC, DHA-PE, and Chol, and 2) DPPC, DLiPC, and Chol, with a small sub set of systems of only DPPC and Chol  used as controls. 

\subsubsection{Simulations}

Martini determined and maintained secondary structures. Ternary structures were initially maintained using Gromac's \cite{grom} position restrains at 1000 $kJ\cdot mol^{-1}$. Later elastic restraints based on Martini's \cite{martini} ElNeDyn algorithm \cite{Periole_Combining_2009} were implemented. Elastic restraints were held at 750 $kJ\cdot mol^{-1}$ using cutoff lengths between 1.0 nm to 1.5 nm. Measuring the rmsd of nAChR's transmembrane back bone averaged at $\sim 2.5$ \r{A}.  

%%Liam Translate gromacs mdp "code" to real words

Molecular dynamics were carried out using the Martini 2.2 force field \cite{martini} and Gromacs 5.0.6 \cite{grom}. Energy minimization was performed at 0.001 $ps$ using 10000 steps. However most energy minimizations finished within $\sim$ 1700 to 3000 steps. Molecular dynamics were run using a time step 0.025 $ps$ for 2 $\mu$s. Simulations used  NPT ensembles. We used Berendsen thermostat with an isotropic pressure couple. The reference temperature was set to 323 Kelvin with temperature coupling constant set to  1 $ps$. The system's compressibility is set to $3e^{-5}$ $bar^{-1}$ with a pressure coupling constant set to 3.0 $ps$. All systems were run using van der Waals (vdW) and Electrostatics in shifted form with a dielectric constant of 15. vdW cutoff lengths were between 0.9 and 1.2 nm, with electrostatic cutoff lengths between 0.0 and 1.2 nm.

The majority of simulations have a box size between $22x22x20$ $nm^3$ and $25x25x25$ $nm^3$, run with Gromacs 5.0.6 \cite{grom}. Systems on the membrane size of $75x75x35$ $nm^3$ were run with Gromacs 5.1.2 \cite{grom}. The z dimension does not vary significantly from simulation box sizes to reduce the number of water beads and decrease computational expense. 

nAChR is held under position restraints in simulations, unless stated other wise.

\subsubsection{Analysis}

\newcommand{\bsat}{b_{\mathrm{sat}}}
\newcommand{\qsat}{Q_{\mathrm{sat}}}
\newcommand{\xsat}{x_{\mathrm{sat}}}
\newcommand{\nbound}{b_{\mathrm{tot}}}
\newcommand{\lo}{l_{\mathrm{o}}}
\newcommand{\ldo}{l_{\mathrm{do}}}

Extent of domain formation within the membrane was tracked by 
    \begin{equation}
    \begin{aligned}
      M_{A, B} = \frac{\langle \eta_{A,B} \rangle} {x_{B}} - 1
    \end{aligned}
    \label{eq:M}
  \end{equation}
 where $\eta_{A,B}$ is the fraction of the 6 nearest neighbors around a given molecule of species type A, that are of species type B, and the average is over time and all molecules of type $A$. For a random mixture, $\langle \eta_{A,B} \rangle = x_{B}$, where $x_{B}$ is the fraction of overall bulk lipids that are of type B. $M_{A,B}>0$ indicates demixing while $M_{A,B}<0$.  

%	$M_{a,b}$ compares measured and expected mixing, where $a$ and $b$ represent a reference and a local lipid species respectively, equation \ref{eq:M}. 
%    \begin{equation}
%    \begin{aligned}
%      M_{a,b} = \frac{\langle \eta_{a,b} \rangle} {\langle \eta_{a,b} \rangle_{rand}} - 1
%    \end{aligned}
%    \label{eq:M}
%  \end{equation}
%  We define $\eta_{a,b}$ as the percentage of lipid species $b$ in contact with lipid species $a$. $M_{a,b}$ is subtracted by one to include all points.

Extent of receptor partitioning within the $\lo$ or $\ldo$ domain was tracked by counting the number $\bsat$ of saturated boundary lipids and comparing with the expectation for a random mixture, via the order parameter $\qsat$:
  \begin{equation}
    \begin{aligned}
      \qsat\equiv \frac{1}{\xsat}\left\langle\frac{  \bsat }{\nbound }\right\rangle-1,\\
    \end{aligned}
    \label{eq:Q}
  \end{equation}
  where $\nbound$ is the total number of lipids in the boundary region and $\xsat$ is the fraction of overall bulk lipids that are saturated phospholipids. $\qsat <0$ indicates depletion of saturated lipids among boundary lipids, as expected for partitioning into an $\ldo$ phase, while $\qsat>1$ indicates enrichment and likely partitioning into an $\lo$ phase. Each frame, $\nbound$ and $\bsat$ were calculated by counting the number of total and saturated lipids, respectively, for which the phosphate bead fell within a distance of 10~\AA~ to 35~\AA~ from the M2 helices, projected onto the membrane plane. 
  
  Two-dimensional distribution of a specific lipid of species $B$ around the protein was calculated using both Cartesian and Polar bins: %$\rho_a$ (\r{A}$^{-2}$) is the density of lipid $a$ within a given bin equations \ref{eq:R}.
  \begin{equation}
    \begin{aligned}
      \rho_{B,i}=\left\langle \frac{n_{B,i}}{A_{i}}\right\rangle \\        
    \end{aligned}
    \label{eq:R}
  \end{equation}
  where $n_B,i$ is the number of lipid species $B$ found within a given bin $i$ and $A$ is the area of a bin. In the case of Cartesian bins, $A = \Delta{x} \Delta{y}$ where the bin widths $\Delta{x} = \Delta{y} = 10$\AA, while for Polar bins, $A = r \Delta{r}\Delta{\theta}$ where $r$ is the projected distance of the bin center from the protein center, $\Delta{r}$= 10\AA~ and $\Delta{\theta} = \frac{\pi}{5}$ radians.
  
%  $Q$ is defined as a metric given by equation \ref{eq:Q} to measure the boundary lipid concentration of a specific lipid species; in this case DPPC. $Q$ can be values between -1 to 1; -1 species $a$ is depleted, 0 species $a$ is randomly mixed, and 1 species $a$ is enriched within the boundary lipids.  
%  \begin{equation}
%    \begin{aligned}
%      Q= \langle\frac{ B_{DPPC}}{N_B \cdot x_{DPPC}}\rangle-1\\
%    \end{aligned}
%    \label{eq:Q}
%  \end{equation}
%  Lipids were measure within a ring of $r>10$ \r{A}   but $r<35$ \r{A}   from the M2 helices. $B_{DPPC}$ is the number of DPPC within ring, where $N_B$ is the total number of boundary lipids. $x_{DPPC}$ is the expected fraction of DPPC if the membrane was randomly mixed. $Q$ is subtracted by one to adjust the metric range from -1 to 1.
%
%	$M_{a,b}$ compares measured and expected mixing, where $a$ and $b$ represent a reference and a local lipid species respectively, equation \ref{eq:M}. 
%    \begin{equation}
%    \begin{aligned}
%      M_{a,b} = \frac{\langle \eta_{a,b} \rangle} {\langle \eta_{a,b} \rangle_{rand}} - 1
%    \end{aligned}
%    \label{eq:M}
%  \end{equation}
%  We define $\eta_{a,b}$ as the percentage of lipid species $b$ in contact with lipid species $a$. $M_{a,b}$ is subtracted by one to include all points.
%
%	$\rho_a$ (\r{A}$^{-2}$) is the density of lipid $a$ within a given bin equations \ref{eq:R}.
%  \begin{equation}
%    \begin{aligned}
%      \rho_{a}=\langle \frac{n_{a}}{A}\rangle \\        
%    \end{aligned}
%    \label{eq:R}
%  \end{equation}
%  $n_a$ is the number of lipid species $a$ found within a given bin and $A$ is the area of a bin. Data was collected with $dx$, $dy$ both equal to 10\r{A} bins for Cartesian coordinates and $dr$ and $d \theta$ as 10\r{A} and $\frac{\pi}{5}$ radians respectively for Polar coordinates.
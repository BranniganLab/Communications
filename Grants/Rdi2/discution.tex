nAChR has a preference for domain interfaces, with the proteins bulk in $l_{do}$ domains. nAChR's $\beta$ subunit has a clear preference for long chained PUFA, while the $\alpha$ subunits seem to favor the cholesterol enriched domains, resulting in  nAChR to saddle the domain domain interface. These results are encouraging when compared to Unwin 2017 \cite{Zuber_Structure_2013}, showing cholesterol rich domains near the $\alpha_{\delta}$-$\delta$ subunits. 

We believe that nAChR partitions itself near the interface to have a source of Chol, as Chol is required for functionality in reconstituted membranes \cite{Fong_Correlation_1986,Sunshine_Lipid_1992,Hamouda_Assessing_2006,Butler_FTIR_1993,Bhushan_Correlation_1993,Fong_Stabilization_1987,Bednarczyk_Transmembrane_2002,Corrie_Lipid_2002}. Comparing \cite{Perillo_Transbilayer_2016} results to our data may suggest they found nAChR partitioned at domain interfaces rather than the $l_o$ domain. Using the lipid species POPC may have created a highly diffuse $l_{do}$ phase providing an abundance of phase interfaces.

%Replacing one zwitterionic head groups for another shows no alteration in membrane partitioning. This suggests nAChR is dependent on acyl chains not zwitterionic lipid head groups. Anionic lipids have shown improved nAChR functionality \cite{Butler_FTIR_1993,Bhushan_Correlation_1993,Fong_Stabilization_1987,Bednarczyk_Transmembrane_2002,Corrie_Lipid_2002}. It is suggested these negatively charged lipids form the boundary domain for nAChR and draw in cations. This research left out anionic lipids in favor of studying the effects of the elevated levels of PE and PUFAs often ignored in both experimentation and computational research.  However, anionic lipids must be included in future simulations to better ascertain if nAChR has a greater preference for PUFAs or charged head groups.

%The well defined boundaries found in systems containing DHA-PE are also observed in systems containing DHA-PC. Shorter acyl chains and greater saturation did not promote well defined domains as seen using DHA (see Figure \ref{fig:fig1}A).  DHA has been shown to stabilize $l_{do}$ domain formation \cite{Levental_Polyunsaturated_2016}. It may be the case running our simulations for longer time would produce well defined domains for any PUFA \cite{Risselada_The_2008}.

%Domains still form when there are three or more lipid species present and one is a PUFA. nAChR does not appear to inhibit or promote domain formation, though it appears to disrupt the domains formed by DHA acyl chains (See Figure \ref{fig:fig2}A). In Figure \ref{fig:fig2}A, $M_{DHA,DHA}$ measurements are observed to be shifted down the $X_{DHA}$ axis when comparing systems with nAChR. 

Literature suggests anionic lipid head groups have greater specificity for nAChR than zwitterionic head groups; the head group phosphatidic acid has been shown to improve nAChR functionality \cite{Butler_FTIR_1993,Bhushan_Correlation_1993,Fong_Stabilization_1987,Bednarczyk_Transmembrane_2002,Corrie_Lipid_2002}. Anionic lipids must be included in future simulations to better ascertain if nAChR has a greater preference for PUFAs or charged head groups.%We hypothesize some of the domain geometries observed (see Figures \ref{fig:fig1}A \textbf{top} and \ref{fig:SIQ}) are due to the smaller box sizes. These geometries are not inductive of a larger membrane.

%Our analysis did not focus $\gamma$ and $\delta$ subunits do not show a significant individual preference for domain, they do appear to have an affinity towards cholesterol over other used lipids. However, both subunits had lower priority than $\alpha$ and $\beta$ subunits, and require further attention.

It has been suggested \cite{Brannigan_Embedded_2008}, the protein density gaps, Fig \ref{fig:sum} subfigure, in nAChR are potential binding domains for cholesterol. Originally thought to contain water, \cite{Brannigan_Embedded_2008} hypothosized cholesterol filled the gaps to stabilize the protein's structure. While we find cholesterol embedded through out the protein's TMD, PUFAs have greater bias. 

We hypothesize PUFAs flexibility allows them to take on multiple confirmations, assisting with deep non-annular binding. Both DHA and LA are observed to embed in nAChR, though LA appears to not embed as deep as frequently as nAChR. We hypothesize saturated lipids chain's rigidity hinders embedding. 

All simulations, have dealt with di-PUFA, and while these lipid species exist in nature they are far less common than lipids with heterogeneous acyl chains. We have begun to take this into consideration, and have observed nAChR partitioning does not alter (data not shown).  

While we observe $\alpha-/\gamma+$ and $\alpha-/\delta+$ to have greater interaction with the $l_o$ and the $\beta$ subunit to hold a preference for $l_{do}$, we have not hypothesized a mechanism. Future analysis is required, analyzing variations in nAChR's sequence and TMD residue-lipid direct and indirect interaction.

Through coarse grained molecular dynamics simulations, we have presented unique data showing nAChR's interaction with PUFA species commonly found in both synaptic and \textit{Torpedo} electric organ membranes.  

% Lipids, especially PUFAs and cholesterol, are found through out the inter- and intra-subunits of the nAChR TMD. nAChR does not appear to inhibit or promote domain formation in simulations. Domains are observed to form regardless of box size, restraints on the protein, or number of proteins. 
